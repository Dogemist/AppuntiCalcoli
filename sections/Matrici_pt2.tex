\section{Matrici (Metodi di rilassamento classici)}
\label{Matrici (Metodi di rilassamento classici)}

Il concetto dei metodi di rilassamento si basa sulla \textit{decomposizione} di A nella seguente forma: $$A = M - N$$
\noindent Il sistema può essere scritto come $$Mx = Nx + b$$ e partendo da un $x_0$ vettore arbitrario, è possibile costruire le relative iterazioni
$$Mx_{k+1} = Nx_k + b$$
I metodi di rilassamento classici costruiscono M e N basandosi sulle seguenti 3 matrici:\\
\noindent una matrice diagonale costruita con gli elementi diagonali di A
 $$D = 
\begin{bmatrix}
a_{11} & 0 & 0 \\
0 & \dots &  0\\
0 & 0 & a_{nn} 
\end{bmatrix}
$$

una matrice triangolare inferiore costruita con gli elementi inferiori alla diagonale di A, tutto il resto zero

$$ -E =
\begin{bmatrix}
0 & & & &\\
a_{21} & 0 & & & \\
\dots  & \dots & \ddots & \\
\dots  &  \dots & \dots & \ddots \\
a_{n1} & \dots & \dots & a_{n,n-1} & 0 
\end{bmatrix}
$$

ed infine una matrice triangolare superiore costruita con gli elementi superiori alla diagonale di A, tutto il resto zero

$$ -F =
\begin{bmatrix}
0 & a_{12} & \dots & \dots & a_{1n} \\
 & 0 & \ddots &  & \dots \\
 &  & \ddots & \ddots  & \dots\\
 &  &  & \ddots & a_{n-1,n}\\
 &  &  &  & 0
\end{bmatrix}
$$

Otteniamo in questo modo $A = D-E-F$
\\ \\
Inoltre, ogni metodo ha una sua particolare matrice d'iterazione.

\subsection{Metodo Jacobi}
\label{Metodo Jacobi}
Con questo metodo, le matrici M e N sono definite nel seguente modo:
\begin{itemize}
\item $M = D$
\item $N = E + F$ Prestate attenzione che i valori delle matrici E ed F di default sono \textbf{Negativi} (-E e -F).
\end{itemize}

Dopo aver determinato M e N si può calcolare la matrice di iterazione $B_j$. Questa può essere calcolata in due modi:
\subsubsection{Calcolo diretto}
$$B_J =
\begin{bmatrix}
0 & -\frac{a_{12}}{a_{11}} & \dots & \dots & -\frac{a_{1n}}{a_{11}} \\
-\frac{a_{21}}{a_{22}} & 0 &  \ddots & & \vdots \\
\vdots & \ddots & \ddots & \ddots & \vdots \\
-\frac{a_{n1}}{a_{nn}} & \dots & \dots & -\frac{a_{n,n-1}}{a_{nn}} & 0
\end{bmatrix}
$$

\subsubsection{Calcolo con D,E,F}
\label{Calcolo con D,E,F}
$$ \boldsymbol{B_J} = M^{-1}*N = \boldsymbol{D^{-1}*(E+F)} $$
dove  $$D^{-1} = 
\begin{bmatrix}
\frac{1}{a_{11}} & 0 & 0 \\
0 & \dots &  0\\
0 & 0 & \frac{1}{a_{nn}}
\end{bmatrix}
$$

\subsubsection{Calcolo dell'iterata successiva}
\label{Calcolo dell'iterata successiva}
Per calcolare $x_{k+1}$ a partire da un $x_k$ arbitrario, basta eseguire il seguente conto $$x_{k+1} = B_J * x_k + q$$
dove $q = D^{-1}*b$

\subsection{Metodo Gauss-Seidel}
\label{Metodo Gauss-Seidel}
Con questo metodo, le matrici M e N sono definite nel seguente modo:
\begin{itemize}
\item $M = D - E$
\item $N = F$
\end{itemize}

In questo caso per calcolare la matrice d'iterazione c'è solo un modo: 
$$B_G = M^{-1}*N = (D-E)^{-1}*F $$
\noindent
\textbf{PRO TIP}: per calcolare l'inversa di M in questo caso si può usare il metodo descritto sul paragrafo \ref{Calcolo dell'inversa} ma essendo lungo ed a tratti macchinoso può portare ad errori. Fortunatamente, essendo una triangolare, è possibile sfruttare questo trick:
\\
\\
\noindent
Una proprietà della matrice identità è che, data una matrice A, $A^{-1}A = I$.
Prendendo in questo caso $(D-E) =
\begin{bmatrix}
1/2 & 1 & 0 \\
-2 & 3 & 0 \\
1 & -1 & 1 
\end{bmatrix}
$ 
\\ \\ \\
\noindent
Abbiamo che $(D-E)^{-1}*(D-E) = I$
\begin{center}
$
\begin{bmatrix}
2 & 0 & 0 \\
a & 1/3 & 0 \\
b & c & 1 
\end{bmatrix}
$
x
$
\begin{bmatrix}
1/2 & 1 & 0 \\
-2 & 3 & 0 \\
1 & -1 & 1 
\end{bmatrix}
$
=
$
\begin{bmatrix}
1 & 0 & 0 \\
0 & 1 & 0 \\
0 & 0 & 1
\end{bmatrix}
$
\end{center}
\noindent
Moltiplico le due matrici ed ottengo: \\
\begin{center}
$
\begin{bmatrix}
1 & 0 & 0 \\
a/2 - 2/3 & 1 & 0 \\
b/2 - 2c+1 & 3c+1 & 1 
\end{bmatrix}
$
=
$
\begin{bmatrix}
1 & 0 & 0 \\
0 & 1 & 0 \\
0 & 0 & 1
\end{bmatrix}
$
\end{center}
\noindent
Come si può notare, ora basta solo far corrispondere gli spazi con le incognite ai relativi valori (in questo caso 0). Per risolvere questo basta risolvere il seguente sistema:
$$\left\{
  \begin{array}{lr}
    a/2 -2/3 = 0 \\
    b/2 -2c+1 = 0 \\
    3c+1 = 0 
  \end{array}
\right.
$$

Ossia rendere $a_{21} = I_{21} $ , $a_{31} = I_{31}  $, $a_{32} = I_{32} $, alla fine sostituiamo i valori di $a,b$ e $c$ nelle rispettive posizioni ed è fatta.
\\ \\
\noindent
\subsubsection{Calcolo dell'iterata successiva gauss-seidel}
\label{Calcolo dell'iterata successiva gauss-seidel}
Per calcolare $x_{k+1}$ a partire da un $x_k$ arbitrario, basta eseguire il seguente conto $$x_{k+1} = B_G * x_k + q$$
dove $q = (D-E)^{-1}*b$
